\section{Implémentation}
\subsection{présentation du jeu}
Comme dit précédemment, notre jeu est un jeu de type puzzle en deux dimensions en vue du dessus. Notre jeu compte 25 niveaux de  base, il est possible de créer nos propres niveau et de l'inclure dans 
notre liste de niveaux. Pour terminer un niveau (et donc finir le puzzle), il faut mettre toutes les caisses sur une platforme, les platformes ne sont pas spécifique à une caisse.

\subsection{Présentation de l' UI}
\subsubsection{écran titre}
	\begin{figure}[h]
	  \centering
	  \includegraphics[width=0.5\textwidth] {pictures/title_screen.png}
	  \caption{écran titre}
	  \label{fig:title_screen}
	\end{figure}
Voici à quoi ressemble l'écran titre de notre jeu, comme vous pouvez le remarquer, l'écran titre est composé du nom de l'application (écrit en alphabet latin et une fois en japonais). Une image de 
présentation du jeu est également présente.

\subsubsection{menu principal}
	\begin{figure}[h]
	  \centering
	  \includegraphics[width=0.5\textwidth] {pictures/main_menu.png}
	  \caption{menu principal}
	  \label{fig:main_menu}
	\end{figure}
Notre menu principal est composé de trois bouttons principaux, un permettant de jouer, l'autre permet d'accéder au menu "paused" et pour terminer le troisième permet de quitter le jeu. L'image de 
fond reste toujours la même qu'à l'écran titre.

\newpage
\subsubsection{menu pause}
	\begin{figure}[h]
	  \centering
	  \includegraphics[width=0.3\textwidth] {pictures/paused_menu.png}
	  \caption{menu pause}
	  \label{fig:paused_menu}
	\end{figure}
Les deux premiers bouttons permettent de régler le volume de la musique et le volume des effets sonores. Pour modifier les valeurs, sélectionné le bouton correspondant grâce au flèche haut et bas du 
clavier. Appuyier sur la touche "enter" et appuyier sur la flèche du haut pour monter le volume ou celle du bas pour le descendre. Une fois le modification faite appuyier sur "esc" pour sortir de la
sélection du boutton. Le dernier boutton permet de reprendre le jeux où il  avait été laissé.

\subsubsection{jeu}
	\begin{figure}[h]
	  \centering
	  \includegraphics[width=0.7\textwidth] {pictures/jeu.png}
	  \caption{jeu}
	  \label{fig:jeu}
	\end{figure}
Voici à quoi ressemble notre jeu. La vue est composé d'un fond à pois (vert sur l'image), de murs qui définise la limite de la zone du puzzle (beige sur l'image), de platformes qui définissent 
l'emplacement de l'objectif (blanc sur l'image) et de caisse (grise sur l'image )qui doivent être déplacer vers les objectifs (blanc sur l'image). les couleurs changent de manière aléatoire pour
chaque reset de niveau.
