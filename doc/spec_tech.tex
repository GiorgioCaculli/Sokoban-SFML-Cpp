\section{Spécification technique}

\subsection{GUI : SFML}
\begin{figure}[h]
	\centering
	\includegraphics{SFML_logo.png}
	\caption{Logo SFML}
	\label{fig:logo_sfml}
\end{figure}
SFML est une librarie qui donne accès à une vaste variétés de fonctionnalités purement écrites en C++. Les cinq fonctionnalités dont on dispose sont les suivantes:
\begin{itemize}
	\item La gestion de toute interaction avec le système d'exploitation
	\item La gestion de fenêtrage
	\item La gestion des graphismes
	\item La gestion du son
	\item La gestion du réseau
\end{itemize}

SFML permet le cross-platforming, soi-disant, un logiciel codé avec SFML aura le même visuel indépadamment du système d'exploitation sur lequel le logiciel tourne.

\subsection{Librairies}
\begin{figure}[h]
	\centering
	\includegraphics[width=0.5\textwidth]{Boost_logo.png}
	\caption{Logo Boost}
	\label{fig:logo_boost}
\end{figure}
La libriairie de logging que nous utiliserons se nomme Boost. En quelque mots, la librarie Boost est elle-même un ensemble de librairies permattant d'étendre les fonctionnalités de C++. Dans notre cas, nous utiliserons les fonctionnalités prédéfinies de Boost.Log, qui nous accès à la possibilités d'enregistrer les différentes intéractions qui ont eu lieu lors de l'exécution du jeu.

\subsection{Fichiers}
Les niveaux et les sauvegardes seront stockées dans des fichiers purement textuels. Ces fichiers ne stockeront que le design des niveau ou de la partie en cours.
Comme déclaré précédemment, Boost est un ensemble de librairies, dans cet ensemble il existe la libairie Boost.JSON. Grâce à cette librairie, nous serons capable de stockers des informations en format JSON, comme par exemple, une liste des scores.

\subsection{OS}
Les systèmes d'exploitation sur lesquels nous testerons notre jeu sont les suivants:
\begin{itemize}
	\item Linux
	\item Mac OSX
	\item MS Windows 10
\end{itemize}
Étant donné que toutes les librairies citées précédemment sont cross-platform, nous serons capables d'assurer que notre jeu est entièrement cross-platform.
