\section{Spécification technique}

\subsection{GUI : SFML}
\begin{figure}[h]
	\centering
	\includegraphics{SFML_logo.png}
	\caption{Logo SFML}
	\label{fig:logo_sfml}
\end{figure}
SFML est une librairie qui donne accès à une vaste variété de fonctionnalités purement écrites en C++. Les cinq fonctionnalités dont nous disposons sont les gestions suivantes:
\begin{itemize}
	\item Toute interaction avec le système d'exploitation
	\item Fenêtrage
	\item Graphismes
	\item Son
	\item Réseau
\end{itemize}

SFML permet le cross-platforming, soi-disant, un logiciel codé avec SFML aura le même visuel indépendamment du système d'exploitation sur lequel le jeu tourne.

\subsection{Librairies}
\begin{figure}[h]
	\centering
	\includegraphics[width=0.5\textwidth]{Boost_logo.png}
	\caption{Logo Boost}
	\label{fig:logo_boost}
\end{figure}
La librairie de logging que nous utiliserons se nomme Boost. En quelques mots, la librairie Boost est elle-même un ensemble de librairies permettant d'étendre les fonctionnalités de C++. Dans notre cas, nous utiliserons les fonctionnalités prédéfinies de Boost.Log, qui nous donne accès à la possibilité d'enregistrer les différentes interactions qui ont eu lieu lors de l'exécution du jeu.

\subsection{Fichiers}
Les niveaux et les sauvegardes seront stockés dans des fichiers purement textuels. Ces fichiers ne stockeront que le design des niveaux ou de la partie en cours.
Comme déclaré précédemment, Boost est un ensemble de librairies, dans cet ensemble il existe la librairie Boost.JSON. Grâce à cette librairie, nous serons capable de stocker des informations en format JSON, comme par exemple, une liste des scores.

\subsection{OS}
Les systèmes d'exploitation sur lesquels nous testerons notre jeu sont les suivants:
\begin{itemize}
	\item Linux
	\item Mac OSX
	\item MS Windows 10
\end{itemize}
Étant donné que toutes les librairies citées précédemment sont cross-platform, nous pouvons assurer que notre jeu sera entièrement cross-platform.
